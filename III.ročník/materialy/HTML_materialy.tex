\documentclass[12pt,a4paper]{article}
\usepackage[utf8]{inputenc}
\usepackage[slovak]{babel}
\usepackage{amsmath, amssymb}
\usepackage{listings}
\usepackage{xcolor}

\lstset{
	basicstyle=\ttfamily\small,
	keywordstyle=\color{blue},
	commentstyle=\color{gray},
	stringstyle=\color{teal},
	showstringspaces=false,
	frame=single,
	breaklines=true
}

\title{Čo je HTML?}
\author{III.ročník}
\date{}

\begin{document}
	\maketitle
	
	
	
	\section{HTML}
		
	Začnime tým, že sa naučíme niečo viac o HTML: čo to presne je a ako ho používame na vytváranie webových stránok. \\
	
	Bez ohľadu na to, ktorý prehliadač uprednostňujete – Chrome, Safari alebo Brave – všetky vykonávajú rovnakú funkciu. Berú rôzne súbory, ako napríklad HTML, CSS a JavaScript, a vykresľujú ich do webovej stránky. \\
	
	Hoci väčšina moderných webových stránok je vytvorená pomocou kombinácie týchto troch typov súborov, webovú stránku nemôžete vytvoriť iba so súborom CSS alebo iba so súborom JavaScript. Môžete však mať iba súbor HTML. V skutočnosti presne takto boli vytvorené prvé webové stránky – s HTML. \\
	
	HTML definuje obsah a štruktúru webovej stránky. Napríklad v tomto prípade súbor HTML vykreslí nadpis s textom „Moja webová stránka“. Keď prehliadač otvorí tento súbor, zobrazí sa tento nadpis. \\
	
	\textbf{Čo presne je HTML?}
	
	Poďme si to rozobrať. HTML je skratka pre Hypertext Markup Language (Hypertext Markup Language).\\
	
	\textbf{Hypertext}
	
	Čo znamená prvá časť, Hypertext? Vzťahuje sa na časti textu, ktoré môžu odkazovať na iné dokumenty v rámci webovej stránky. Tieto časti textu sú hypertext alebo hypertextové odkazy a tvoria základ fungovania webovej stránky HTML.\\
	
	Ak preskúmame prvú webovú stránku na svete, ktorú vytvoril Sir Tim Berners-Lee, ktorý tiež vynašiel internet, vidíme, že je vyplnená hypertextovými odkazmi označenými modrou farbou. Ako sa očakávalo, kliknutím na ktorýkoľvek z týchto hypertextových odkazov sa dostanete do iného dokumentu, do iného súboru HTML.\\
	
	Napríklad, ak by sme boli na Project.html a klikli by sme na hypertextový odkaz typu „Ako môžem pomôcť?“, presmeroval by nás do iného súboru HTML. Takto funguje hypertext. Na tejto webovej stránke je dokonca odkaz, ktorý presne vysvetľuje, čo je hypertext.\\
	
	\textbf{Značkovací jazyk}
	
	Teraz, keď rozumieme hypertextu, čo s druhou časťou: Značkovacím jazykom? Čo je to značkovací jazyk? \\
	
	V anglickom jazyku často vidíme malé časti, ktoré fungujú podobne ako značky, ako napríklad dvojité úvodzovky. Prítomnosť týchto úvodzoviek čitateľovi hovorí, že táto časť je citát. \\
	
	Je to podobné tomu, čo nájdete v redaktorskej recenzii rukopisu, kde sa značky používajú na zobrazenie rôznych vecí, napríklad ktoré časti je potrebné zvýrazniť tučným písmom pridaním vlnovky pod ne alebo ktoré časti je potrebné podčiarknuť pridaním rovnej čiary.\\
	
	\textbf{Ako sa robí značkovanie v HTML?}
	
	Značky v HTML sa vykonávajú pomocou takzvaných HTML tagov. \\
	
	V raných dobách internetu existovalo len niekoľko HTML tagov. Dnes ich je oveľa viac, ale realisticky budete používať iba niektoré z najdôležitejších, ako sú nadpisy (h1 až h6) a tag odseku (p).\\
	
	Tu vidíte všetky HTML tagy, ale keď ich zúžime na tie, ktoré skutočne potrebujete poznať, stanú sa oveľa prístupnejšími.\\
	
	Vo zvyšku kurzu sa budeme venovať niektorým z najdôležitejších tagov, takže sa nemusíte báť zapamätať si ich alebo sa ich všetky naraz naučiť. Budeme ich preberať podľa potreby v našich projektoch a cvičeniach.\\
	
	V ďalšej lekcii sa začneme o značke nadpisu (h), jednej z najoriginálnejších značiek HTML, aké kedy existovali. O tom všetkom a ešte viac sa dozvieme v ďalšej lekcii. \\
	
	\newpage
	\textbf{Kľúčové poznatky} 
	\begin{itemize}
		\item HTML je skratka pre Hypertext Markup Language (Hypertext Markup Language) a je základným jazykom na vytváranie webových stránok, 
		\item Hypertext označuje text, ktorý odkazuje na iné dokumenty a tvorí základ navigácie na webových stránkach, 
		\item Značkovací jazyk používa značky na definovanie štruktúry a prezentácie obsahu, podobne ako redakčné značky v rukopisoch,
		\item Medzi základné značky HTML patria nadpisy (h1 až h6) a odseky (p), ktoré budú v kurze postupne preberané.
	\end{itemize} 
	
	
\end{document}
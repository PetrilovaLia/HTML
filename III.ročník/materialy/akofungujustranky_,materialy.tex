\documentclass[12pt,a4paper]{article}
\usepackage[utf8]{inputenc}
\usepackage[slovak]{babel}
\usepackage{amsmath, amssymb}
\usepackage{listings}
\usepackage{xcolor}

\lstset{
	basicstyle=\ttfamily\small,
	keywordstyle=\color{blue},
	commentstyle=\color{gray},
	stringstyle=\color{teal},
	showstringspaces=false,
	frame=single,
	breaklines=true
}

\title{Ako webové stránky v skutočnosti fungujú?}
\author{III.ročník}
\date{}

\begin{document}
	\maketitle
	
	
	
	\section{	Úvod do fungovania webových stránok}
	
	Na tejto hodine sa dozvieme, ako webové stránky v skutočnosti fungujú, a pochopíme úlohu prehliadača a rôznych súborov, ktoré používa na zobrazenie vašich obľúbených webových stránok. \\
	
	Predtým ste sa dozvedeli, že internet pozostáva z kábla, ktorý spája klientske počítače so serverovými počítačmi. Existujú špeciálne druhy serverových počítačov nazývané servery Domain Name Service (DNS), ktoré fungujú ako veľký telefónny zoznam Zlatých stránok a dokážu vyhľadať IP adresu ľubovoľnej webovej stránky, ku ktorej chcete pristupovať. \\
	
	Keď zistíte túto IP adresu, môžete priamo kontaktovať serverový počítač webovej stránky, ktorú chcete zobraziť, a ten vám odošle všetky súbory a údaje, ktoré váš prehliadač potrebuje na zobrazenie na obrazovke. \\
	
	Údaje, ktoré z tohto servera dostanete, zvyčajne pozostávajú z troch typov súborov: HTML, CSS a JavaScript. Tieto sú veľmi bežné a sú neoddeliteľnou súčasťou fungovania webových stránok. \\
	
	Ale čo presne robia a prečo existuje toľko rôznych typov súborov? Prečo nemôžeme mať len jeden súbor, ktorý predstavuje celú webovú stránku? V skutočnosti majú veľmi odlišné úlohy. \\
	
	\textbf{HTML: Obsah webovej stránky}
	
	Súbor s kódom HTML je zodpovedný za obsah vašej webovej stránky. Ak by webová stránka bola domom, potom by HTML boli skutočné tehly domu. Je to surový materiál, z ktorého je váš dom vyrobený. Podobne súbor HTML obsahuje obsah vašej webovej stránky, ako je textový obsah, obrázky, tlačidlá alebo odkazy.\\
	
	\textbf{CSS: Štýlovanie vašej webovej stránky}
	
	Druhým typom súborov sú súbory CSS, ktoré sú zodpovedné za štýlovanie vašej webovej stránky, ako je farba stien alebo tvar vašich dverí. Súbor CSS určuje, ako bude vaša webová stránka vyzerať. Napríklad rozhoduje o farbe pozadia stránky alebo o tom, či budú mať tlačidlá zaoblené rohy. Zameriava sa na všetok obsah na vašej webovej stránke vytvorený pomocou HTML a na tieto prvky aplikuje štýl. Môžete použiť CSS na vytvorenie červeného tlačidla so zaoblenými rohmi a nastaviť text tlačidla na určité písmo.\\
	
	\textbf{JavaScript: Pridávanie funkcií}
	
	Poslednou súčasťou je kód JavaScript, ktorý umožňuje vašej webovej stránke skutočne robiť veci alebo mať funkčnosť. Ak staviate dom, bolo by to ako pridanie žiaroviek, ktoré sa dajú zapínať a vypínať, alebo inštalácia sporáka, ktorý dokáže zapáliť oheň na zohriatie jedla. Váš dom sa tak premení na domov a JavaScript robí presne to isté pre webovú stránku. Statickú webovú stránku, ktorá má len pekné obrázky alebo text, premení na niečo, s čím môže používateľ skutočne interagovať. Napríklad odoslanie e-mailu v Gmaile alebo zverejnenie raňajok na Instagrame. Umožňuje to vašej webovej stránke stať sa funkčnou, nielen niečím pekným na pohľad. \\
	
	\textbf{Príklad: Domovská stránka Google}
	
	Keď si vezmeme ako príklad domovskú stránku Google, akonáhle dostaneme tieto tri typy súborov zo servera Google, náš prehliadač, čo je softvér špecializujúci sa na prácu s týmito súbormi, ich načíta. Keď prehliadač načíta HTML, vidíme obsah webovej stránky, ako napríklad obrázok loga Google, dve tlačidlá a textové pole, do ktorého môžeme zadať hľadaný výraz. \\
	
	Keď prehliadač načíta súbory CSS, upraví vzhľad týchto komponentov. Pomocou CSS nezískame žiadne ďalšie tlačidlá ani obrázky, ale webová stránka vyzerá presne tak, ako to Google zamýšľal, napríklad tvar textového poľa alebo farba tlačidiel. \\
	
	Nakoniec, pomocou súboru JavaScript prehliadač poskytuje funkčnosť webovej stránky. Napríklad môžeme zadať vyhľadávací výraz ako „Google v roku 1998“. Keď stlačíte tlačidlo vyhľadávania, zobrazí sa Google v podobe, v akej vyzeral v roku 1998. Pomocou týchto troch rôznych súborov získame obsah, štýl a funkčnosť webovej stránky, ktoré skombinujeme, a vytvoríme tak moderné webové stránky. \\
	
	\textbf{Kontrola webových stránok pomocou nástrojov pre vývojárov prehliadača Chrome}
	
	S týmito vedomosťami môžeme začať experimentovať so skutočnými webovými stránkami na internete. Ak otvoríte prehliadač a prejdete na google.com, môžete kliknúť pravým tlačidlom myši na tlačidlo „Vyhľadávanie Google“ a kliknúť na „Kontrola“. Otvorí sa Nástroje pre vývojárov prehliadača Chrome, ktoré sú jedným z najlepších balíkov nástrojov pre webových vývojárov. Preto sa odporúča používať prehliadač 
	
	\textbf{Chrome na vývoj webových stránok.}
	
	Nástroje pre vývojárov prehliadača Chrome automaticky zvýraznia časť kódu zodpovednú za tlačidlo, ktoré ste skontrolovali. Napríklad v kóde sa zobrazuje názov tlačidla s názvom „Vyhľadávanie Google“. Môžete naň dvakrát kliknúť a upraviť tento názov tak, aby znel úplne inak, napríklad „Vyhľadávanie Angela“. Po stlačení klávesu Enter sa zmena na stránke aktualizuje. \\
	
	Existuje aj atribút „aria-label“ s hodnotou „Vyhľadávanie Google“, ktorý sa používa iba pre čítačky textu, a nie pre zobrazenie v prehliadači. Aj tento atribút môžete zmeniť, ale aktualizuje niečo v zákulisí a nie je viditeľný na obrazovke. Pri úprave nezabudnite dvakrát kliknúť na správny atribút. \\
	
	V závislosti od obsahu HTML, ktorý kontrolujete, sa časť, ktorú potrebujete zmeniť, môže líšiť. Napríklad, ak chcete zmeniť titulky na techcrunch.com, môžete kliknúť pravým tlačidlom myši na titulok, kliknúť na „Kontrola“, nájsť príslušný kód a dvojitým kliknutím ho upraviť. Môžete zmeniť titulnú stranu TechCrunch, BBC News alebo akejkoľvek webovej stránky, ktorú chcete. Je to zábavný spôsob, ako si urobiť žart z priateľov zmenou viditeľného textu na webových stránkach, ktoré navštívia. \\
	
	Po obnovení webovej stránky sa však všetko obnoví do pôvodnej verzie. Je to preto, že obnovenie požiada server o opätovné doručenie súborov HTML, CSS a JavaScript na vykreslenie webovej stránky. Zmeny vykonané v nástrojoch pre vývojárov Chrome ovplyvňujú iba vašu lokálnu verziu a pri obnovení sa neuložia. \\
	
	V nasledujúcich lekciách budeme pracovať s HTML, CSS a JavaScriptom na vytváraní a hosťovaní vašich vlastných webových stránok naživo na internete. Naučením sa, ako kódovať a vytvárať webové stránky, budete schopní vytvárať webové stránky, ktoré hovoria čokoľvek chcete, vyzerajú tak, ako chcete, a majú funkcie, ktoré potrebujete. \\
	

	
	
	\textbf{Kľúčové poznatky} 
	\begin{itemize}
		\item Webové stránky sa skladajú z troch hlavných typov súborov: HTML pre obsah, CSS pre štýl a JavaScript pre funkčnosť, 
		\item HTML poskytuje štruktúru a základné materiály webovej stránky, podobne ako tehly v dome, 
		\item CSS upravuje štýl prvkov webovej stránky a určuje vzhľad, ako sú farby a tvary,
		\item JavaScript pridáva interaktivitu a funkčnosť, čím mení statické stránky na dynamické zážitky.
	\end{itemize} 
	
	
\end{document}
\documentclass[12pt,a4paper]{article}
\usepackage[utf8]{inputenc}
\usepackage[slovak]{babel}
\usepackage{amsmath, amssymb}
\usepackage{listings}
\usepackage{minted} % balík na farebné zvýraznenie kódu
% dostupné štýly: friendly, colorful, monokai, borland, autumn, vs, tango, ...
\usemintedstyle{emacs}
\usepackage[dvipsnames]{xcolor} % example xcolor package option, see the documentation


\title{Prehľad HTML a CSS}
\author{IV.ročník}
\date{}

\begin{document}
	\maketitle
	
	\section{Jazyk HTML}
	\begin{itemize}
		\item hypertextový značkovací jazyk
		\item základným prvkom je \textbf{tag}
		\begin{itemize}
			\color{OliveGreen}
			\item párové tagy
			\item nepárové tagy
			\item nepovinne párové
		\end{itemize}
	\end{itemize}
	
	\textcolor{white}{teolaxt} \\ 
	\textbf{Tag} - základná značka, ktorí symbolizuje pre prehliadač potrebný prvok \\
	\textbf{Atribút} - pomenovanie vlastnosti HTML tagu \\
	\textbf{Hodnota} - konkrétna hodnota priradená atribútu a tagu 
	
	\mint{html}|<tag atribút="hodnota"> Obsah, na ktorý sa aplikujú nastavenia </tag>|

	\section{CSS}
	\begin{itemize}
		\item priraďujú HTML tagom rôzne vlastnosti
		\item definície zapísané 
		\begin{itemize}
			\item zapísaná v HTML dokumente
			\item priradená konkrétnemu atribútu
			\item pripojené externe ako CSS súbor
		\end{itemize}
	\end{itemize}
	
	\newpage
	\section*{Základná štruktúra}
	\begin{figure}[h]
		\begin{minted}[autogobble]{html}
			<!DOCTYPE html>
			<html lang="en">
			<head>
			  <meta charset="UTF-8">
			  <meta name="viewport" content="width=device-width,
			  initial-scale=1.0">
			  <title>Document</title>
			</head>
			<body>
			
			</body>
			</html>
		\end{minted}
	\end{figure}

	\section*{Zápis CSS}
	\begin{figure}[h]
		\begin{minted}[autogobble]{css}
			tag{css-vlastnost: hodnota1;
			    css-vlastnost: hodnota2;
			    }
		\end{minted}
	\end{figure}
	
	\begin{enumerate}
		\item v hlavičke html dokumentu \mint{css}|<style></style>|
		\item priamo pre daný tag ako atribút\mint{css}|<h1 style="color: blue;">|
		\item externý .css súbor \mint{css}|<link rel="stylesheet" href="./style.css"/>|
	\end{enumerate}
	
	\section{Meta tagy}
	Definujú metadáta o HTML dokumente, nachádzajú vo vnútri elementu <head> a na stránke sa nezobrazia, ale sú strojovo analyzovateľné. Metadáta používajú prehliadače (ako zobraziť obsah alebo znova načítať stránku), vyhľadávače (kľúčové slová) a ďalšie webové služby. 
	
	\begin{enumerate}
		\item \textbf{Description}
		\begin{itemize}
			\item obsahuje popis stránky
			\item \mint{html}|<meta name="description" content="HTML a CSS">|
		\end{itemize}
		\item \textbf{Keywords}
		\begin{itemize}
			\item obsahuje kľúčové slová, dnes nemajú význam, Google ich ignoruje
			\item \mint{html}|<meta name="keywords" content="HTML, CSS, JavaScript">|
		\end{itemize}
		\item \textbf{Jazyk}
		\begin{itemize}
			\item pomáha pri vyhľadávaní v danom jazyku
			\item \mint{html}|<meta http-equiv="Content-language" content="sk">|
		\end{itemize}
		\item \textbf{Kódovanie}
		\begin{itemize}
			\item zabezpečuje správne zobrazovanie znakov s diakritikou
			\item \mint{html}|<meta charset="UTF-8">|
		\end{itemize}
		\item \textbf{Autor}
		\begin{itemize}
			\item kto stránku vytvoril
			\item \mint{html}|<meta name="autor" content="Michal Hricko">|
		\end{itemize}
	\end{enumerate}
	
	\newpage
	\section{Triedy a identifikátory}
	
	\subsection*{Triedy}
	Atribút class sa často používa na odkazovanie na názov triedy v štýle, teda uľahčujú dizajnovanie. Môže ho tiež použiť JavaScript na prístup a manipuláciu s \textbf{prvkami} s daným názvom triedy.
	
	\subsubsection*{\textbf{Zápis html}}
	\mint{html}|<p class="note">Ola amigos</p>|
	
	\subsubsection*{\textbf{Zápis css}}
	\mint{css}|.note{font-size: 20px;}|
	
	\subsection*{Identifikátory}
	
	Majú podobné využitie ako triedy, ale v HTML dokumente \textbf{sa nesmú opakovať!}
	
	\subsubsection*{\textbf{Zápis html}}
	\mint{html}|<p id="header"> </p>|
	
	\subsubsection*{\textbf{Zápis css}}
	\mint{css}|#header{font-size: 20px;}|
	
	\subsection*{Pseudotriedy}
	
	Umožňujú zadávanie vlastností pre \textbf{určitý tag} v špecifických prípadoch (definuje štýl pre špeciálny stav prvku). \\
	
	\noindent\textbf{:hover} - Vyberie prvok po prejdení myšou(podčiarkne, zmení farbu), \\
	\textbf{:link} - nenavštívený odkaz, \\
	\textbf{:visited} - už navštívený odkaz, \\
	\textbf{:fullscreen} - zobrazenie na celú obrazovku.
	
	
	
	
\end{document}
